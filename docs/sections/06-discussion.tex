\section{Discussion}
\subsection{Results}
The experiments that have been conducted provide a large amount of insight in whether or not it is beneficial to use V2V and autonomous driving technology. 

As seen in the analysis of the experiments, it can clearly be seen that an increase in proportion of autonomous vehicles lead to an increase in flow and higher average speed for all vehicles. It can be seen from the results in figure \ref{fig:experiment1} of the experiments, how after an 80\% proportion of autonomous vehicles, the flow and speed is nearly the optimal in a simple highway experiment. 

The results also shows that even though not 100\% of vehicles are using the autonomous and V2V communication technology, there is still a significant increase in flow of cars and the average speed of the cars. The increase is actually close to being doubled when the proportion is only 60\% 

This means that it is not necessary for everyone to be using these two technologies for there to be a benefit from it. Because of this the technology can be introduced over a longer period of time as it becomes more available to the general public to use while stile providing benefits for everyone.  

It can also be seen from the experiments in figure \ref{fig:experiment3} how that when the speed limit increases, automation of vehicles becomes even more effective compare to a regular vehicle. 
This allows for cities to increase speed limits, without worrying about accidents and traffic jams which allows people to spend less time on the road.

These results clearly indicate how autonomous vehicles would greatly improve the traffic situation on a highway with recurring traffic problems. However the results does not account for the factors which were not simulated, such as weather, roadwork and other external factors that can also have an effect on the highway. But it still shows how on a very basic level autonomous cars can increase the traffic flow of a simple highway.

