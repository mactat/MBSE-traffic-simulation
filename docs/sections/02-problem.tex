\section{Problem Statement}
\label{sec:problem-statement}
This section is focused on exploring the challenges of reducing and preventing traffic using vehicle to vehicle (V2V) communication and autonomous driving.
\subsection{Phantom Traffic}
Traffic can occur for multiple reason and for this project a specific type, called \textit{phantom traffic}, was considered. Is one of the most common and more complex form of traffic as is not due to something concrete like an accidents or a traffic lights. It occurs when a lot of cars are on the roads at the same time and just a simple disturbance in the flow can create a ripple effect that generates the congestion. Even a common maneuver like lane changing, if poorly executed, can lead to serious traffic jam \cite{Phantom}.

In essence phantom traffic is caused by the imperfection of the human being, by its relatively low reaction time and its limited perception of the space around itself. For this reason it was questioned if in order to solve congestion the solution could be to remove the human factor from the equation and let computers handle decisions on the road.
\subsection{V2V Communication and Car Automation}
An autonomous car by itself could probably help reduce traffic using sensors, cameras and technologies like adaptive cruise control\cite{ACC} but for this project it was decided to address the problem through the additional implementation of a wireless communication network inspired by the Vehicle-to-vehicle (V2V) communication paradigm \cite{V2V}. V2V communication consists of a network specifically designed for vehicles that allows the sharing of real time information including data like velocity, position, acceleration and general status of the car. 

With this kind of information the technological upgrade to an autonomous car capable of making it own decisions based not only to sensors but also on the received information from other cars would not only be easily achievable but would also lead to improve the efficiency of an autonomous car\cite{V2Vimprovement}. 

\subsection{Question}
This project will try to demonstrate that a system that combine V2V communication and car automation would allow each autonomous car to know in advance what is happening further down the road and react with much more efficiency than a human. This paper is going to discuss if, in a scenario with recurrent congestion due to phantom traffic, it is possible to increase vehicle flow and also what proportion of autonomous connected cars would be needed.




